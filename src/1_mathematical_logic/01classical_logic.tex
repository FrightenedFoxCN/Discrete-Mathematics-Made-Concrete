\chapter{古典逻辑学的观念}

\begin{quote}
    Each mortal thing does one thing and the same: \\
    Deals out that being indoors each one dwells; \\
    Selves - goes itself; myself it speaks and spells, \\
    Crying What I do is me: for that I came.
\flushright{G. M. Hopkins \textit{As kingfishers catch fire}}
\end{quote}

古典逻辑学的历史我们已经谈了不少了,接下来我们需要谈谈它实质性的内容. 当然,这些内容几乎都是近乎常识,但是 Rosen 的书中还是把它专划作一个章节讨论. 因此,我们在此也不妨将其单独拎出来作为一讲. 这里主要讨论的是命题、逻辑连接词、量词、谓词和范式的问题,同时,真值表的处理方式也被放在这里. 

\section{命题}

什么是\keyterm{命题}[proposition]?高中的时候我们就知道,命题是一个或真或假的陈述性语句. 比方说,“太阳东升西落”就是一个命题,而“存在一个包含所有不属于自身的集合的集合”就不是一个命题. 当然,后半句话有点拿当朝的剑斩前朝的官的意思,因为这个悖论在古典逻辑盛行的时候并没有出现,但既然 Rosen 的书上这么定义,而且有些老师也会这样举例,那我们就姑且承认这不是一个命题. 

然后,我们用类似语言哲学家们的方式来处理\keyterm{逻辑连结词}[connective]非:

\begin{definition}
    设 $p$ 为一命题,则 $\lnot p$ 表示命题“情况并非如此,$p$”(it is not the case that...). 
\end{definition}

$\lnot p$ 被称为命题 $p$ 的\keyterm{否定}[negation]. “与”和“或”也用类似的方式定义:

\begin{definition}
    设 $p, q$ 为命题,则 $p \wedge q$ 表示 $p$ 且 $q$,即当且仅当 $p$ 和 $q$ 均为真时它才为真;$p \vee q$ 表示 $p$ 或 $q$,即当且仅当 $p$ 和 $q$ 中至少有一个为真时它才为真. 
\end{definition}

$p \wedge q$ 也被称为 $p$ 和 $q$ 的\keyterm{合取}[conjunction],$p \vee q$ 则被对应的称为 $p$ 和 $q$ 的\keyterm{析取}[disjunction]. 单个的命题 $p$ 和 $q$ 被称为\keyterm{原子命题}[atomic proposition],它被认为是不可再分的;而由逻辑连接词连接起来的命题则被称为\keyterm{复合命题}[compound proposition]. 显然,面对实际问题时,原子命题的选择具备某种程度上的任意性. 逻辑连接词又称\keyterm{逻辑运算符}[logical operator]. 

现在看来除了术语已经开始复杂化了以外都很顺利,但是,这种连接词的表达方式多少有点不够形式化. 因此,我们引入\keyterm{真值表}[truth table]的概念. 

\begin{table}[htpb]
    \caption{逻辑非的真值表}
    \begin{center}
    \begin{tabular}{c|c}
    \hline
    $p$ & $\lnot p$ \\ \hline
    $T$ & $F$       \\ \hline
    $F$ & $T$       \\ \hline
    \end{tabular}
    \end{center}
\end{table}

\begin{table}[htpb]
    \caption{逻辑与的真值表}
    \begin{center}
    \begin{tabular}{c|c|c}
    \hline
    $p$ & $q$ & $p \wedge q$ \\ \hline
    $T$ & $T$ & $T$          \\ \hline
    $T$ & $F$ & $F$          \\ \hline
    $F$ & $T$ & $F$          \\ \hline
    $F$ & $F$ & $F$          \\ \hline
    \end{tabular}
    \end{center}
\end{table}

\begin{table}[htpb]
    \caption{逻辑或的真值表}
    \begin{center}
    \begin{tabular}{c|c|c}
    \hline
    $p$ & $q$ & $p \vee q$ \\ \hline
    $T$ & $T$ & $T$          \\ \hline
    $T$ & $F$ & $T$          \\ \hline
    $F$ & $T$ & $T$          \\ \hline
    $F$ & $F$ & $F$          \\ \hline
    \end{tabular}
    \end{center}
\end{table}

其中左边的一列/两列称为对命题的\keyterm{赋值}[assignment/evaluation],后边一列则是逻辑连接符运算的结果,$T$ 表示真,$F$ 表示假. 这样,我们就能定义一些更加复杂的连接词:

\begin{definition}
    设 $p, q$ 为命题,则 $p \oplus q$ 表示 $p$ 和 $q$ 的\keyterm{异或}[exclusive or],它为真当且仅当 $p$ 和 $q$ 中有且仅有一个为真. 
\end{definition}

\begin{table}[htpb]
    \caption{异或的真值表}
    \begin{center}
    \begin{tabular}{c|c|c}
    \hline
    $p$ & $q$ & $p \oplus q$ \\ \hline
    $T$ & $T$ & $F$          \\ \hline
    $T$ & $F$ & $T$          \\ \hline
    $F$ & $T$ & $T$          \\ \hline
    $F$ & $F$ & $F$          \\ \hline
    \end{tabular}
    \end{center}
\end{table}

\begin{definition}
    设 $p, q$ 为命题,则 $p \to q$ 表示若 $p$ 则 $q$,它为真当且仅当 $p$ 为假或者 $p$ 和 $q$ 均为真. 这种式子被称为\keyterm{蕴含式}[implication]或\keyterm{条件句}[conditional statement],其中 $p$ 称为\keyterm{前提}[premise]、\keyterm{假设}[hypothesis]或\keyterm{前项}[antecedent],$q$ 称为\keyterm{结论}[conclusion]或\keyterm{结果}[consequence]. 
\end{definition}

\begin{table}[htpb]
    \caption{蕴含的真值表}
    \begin{center}
    \begin{tabular}{c|c|c}
    \hline
    $p$ & $q$ & $p \to q$ \\ \hline
    $T$ & $T$ & $T$          \\ \hline
    $T$ & $F$ & $F$          \\ \hline
    $F$ & $T$ & $T$          \\ \hline
    $F$ & $F$ & $T$          \\ \hline
    \end{tabular}
    \end{center}
\end{table}

这里有一个有趣的问题:当前提为假的时候,结论不管怎么说都是真的. 这看起来反直觉,但其实是一个很平常的事情,看下面这句话:

\textit{如果你好好学习,你就不会挂科. }

这句话是真的,哪怕你没有好好学习然后挂科了,或者没有好好学习但没挂科. 前提不成立的情况下,任何事实结果都不会影响这个命题是否是真的. 在下面这个对话中,这个性质就表现得更加明显了:

\textit{``Can you v me 50?" ``When pigs fly!"}

第二个说话者的意思事实上是绝无此可能. 因为猪会飞是一个假命题,所以它使得被省略的命题(\textit{``When pigs fly, I can v you 50"})一定成立,但这是反事实的. 这就构成了一个重要的逻辑原则,它被称为 ex falso sequitur quodlibet:从谬误中,你想怎么干就怎么干,形式化地说,对于任意命题 $p$,$F \to p$ 成立. 

最后一个逻辑运算符比较简单:

\begin{definition}
    设 $p, q$ 为命题,则 $p \leftrightarrow q$ 表示 $p$ 当且仅当 $q$,即两者具备相同的真值. 我们将其称为\keyterm{双向蕴含式}[biconditional statements/bi-implications]
\end{definition}

\begin{table}[htpb]
    \caption{双向蕴含的真值表}
    \begin{center}
    \begin{tabular}{c|c|c}
    \hline
    $p$ & $q$ & $p \leftrightarrow q$ \\ \hline
    $T$ & $T$ & $T$          \\ \hline
    $T$ & $F$ & $F$          \\ \hline
    $F$ & $T$ & $F$          \\ \hline
    $F$ & $F$ & $T$          \\ \hline
    \end{tabular}
    \end{center}
\end{table}

我们约定,逻辑运算符的优先级为 $\lnot$ 最高,$\wedge$ 和 $\vee$ 其次,$\to$ 和 $\leftrightarrow$ 最低. 因为 $\oplus$ 基本上不会出现,所以我们也可以预设它的有限集和 $\vee$ 一致,但这没什么影响. 下面几个概念事关命题的真值:

\begin{definition}
    称一个命题是\keyterm{重言式}[tautology]如果它对于任意指派都是真的;称其为\keyterm{矛盾式}[contradiction]如果它对于任意指派都是假的;称其为\keyterm{可能式}[contingency]如果它既不是重言式,也不是矛盾式. 重言式又称永真式,矛盾式又称永假式,重言式和可能式统称为\keyterm{可满足的}[satisfiable]. 
\end{definition}

在这一小节最后,我们讲讲怎么利用真值表推算复合命题的真值,判断其为重言式、矛盾式还是可能式. 下面是一个例子:

\begin{table}[htpb]
    \caption{$(p \vee \lnot q) \to (p \wedge q)$ 的真值表}
    \begin{center}
    \begin{tabular}{c|c|c|c|c|c}
    \hline
    $p$ & $q$ & $\lnot q$ & $p \vee \lnot q$ & $p \wedge q$ & $(p \vee \lnot q) \to (p \wedge q)$ \\ \hline
    $T$ & $T$ & $F$ & $T$ & $T$ & $T$ \\ \hline
    $T$ & $F$ & $T$ & $T$ & $F$ & $F$\\ \hline
    $F$ & $T$ & $F$ & $F$ & $F$ & $T$ \\ \hline
    $F$ & $F$ & $T$ & $T$ & $F$ & $F$ \\ \hline
    \end{tabular}
    \end{center}
\end{table}

也就是说,只要逐级把表达式拆开,然后从小到大计算. 这样的思路在后面讨论数理逻辑的问题也会碰到,这里其实有一个被称为\keyterm{结构归纳}[structural induction]的方法,留待后面才会得到真正意义上的广泛应用. 

\section{命题的等性以及范式}

从这里开始,我们需要考虑两种东西:命题内部的运算,以及命题与命题之间的关系. 前者就是如上所述的连接词,后者则被我们称为``元命题''的连接词. 下面我们要定义第一个元命题的连接词:(语义)等价.

\begin{definition}
    称两个命题 $p$ 和 $q$ 是\keyterm{语义等价}[semantically equivalent]的,如果 $p \leftrightarrow q$ 为重言式,记作 $p \equiv q$,简称\keyterm{等价}[equivalent]. 
\end{definition}

容易验证:$\equiv$ 是一个等价关系,即它具备自反性、对称性和传递性. 这个结论的验证只需观察 $p \leftrightarrow q$ 的真值表即可. 当然,如果不在意命题和元命题的区别,在这里可能就会有点犯迷糊了. 下面我们要给出第一个重要的元命题,作为例子:

\begin{theorem}[de Morgen]
    \[
        \lnot (p \wedge q) \equiv \lnot p \vee \lnot q, \quad \lnot (p \vee q) \equiv \lnot p \wedge \lnot q    
    \]
\end{theorem}

验证类似的定理有两种方法,一种是利用已经建立的元命题,另一种是利用真值表. 鉴于这是我们建立的第一个元命题,我们给出真值表的证明:

\begin{proof}
    我们只证第一个等式:

    \begin{table}[htpb]
        \caption{$\lnot (p \wedge q) \leftrightarrow \lnot p \vee \lnot q$ 的真值表}
        \begin{center}
        \begin{tabular}{c|c|c|c|c|c|c|c}
        \hline
        $p$ & $q$ & $p \wedge q$ & $\lnot (p \wedge q)$ & $\lnot p$ & $\lnot q$ & $\lnot p \vee \lnot q$ & $\lnot (p \wedge q) \leftrightarrow \lnot p \vee \lnot q$ \\ \hline
        $T$ & $T$ & $T$ & $F$ & $F$ & $F$ & $F$ & $T$ \\ \hline
        $T$ & $F$ & $F$ & $T$ & $F$ & $T$ & $T$ & $T$ \\ \hline
        $F$ & $T$ & $F$ & $T$ & $T$ & $F$ & $T$ & $T$ \\ \hline
        $F$ & $F$ & $F$ & $T$ & $T$ & $T$ & $T$ & $T$ \\ \hline
        \end{tabular}
        \end{center}
    \end{table}

    这里我们的证明策略就是把元命题转化成命题. 当然,实际上为了简明起见,包括我们的教科书中,都会直接写出左半边的真值表和右半边的真值表,然后对比发现它们相等,完成证明,读者亦可自由地采取这种方法. 
\end{proof}

第二个定理实际上可以看成一种公理,不过它在我们的语境下还是一个定理. 为了明确表达它的含义,需要用到\keyterm{项}[term]和\keyterm{子项}[subterm]的概念,但是,我们仅给出一个描述,它应当是非常符合直觉的:

\begin{theorem}
    如果命题 $p_1$ 是命题 $p$ 的一部分,且 $p_1 \equiv p_2$,将 $p$ 中的 $p_1$ 替换成 $p_2$ 得到 $p'$,则 $p \equiv p'$. 
\end{theorem}

这个命题的证明用到结构归纳法,在此仅稍作叙述,容后介绍了一些模型论的语言再做详细证明. 我们对 $p$ 的结构做归纳,如果 $p$ 是原子命题,那么 $p_1$ 只能是它自身,由 $\equiv$ 的传递性和自反性得证;如果它是由 $p_1$ 经过某些逻辑连接词连接起来的,分别证明逻辑连接词的左侧、右侧替换之后真值相同,这样按照``从大结构往小结构分拆''的方式,可以得到结果. 

实际上,如果不考虑严格证明,这个事情是非常直观的. 这就相当于,我们改变了一部分的形式而没有改变语义——这就是语义等价这里所谓``语义''的来源,它和``语法''对立,前者指其真值,后者指其在形式上的表达,比方说下面两句话是一样的:

\textit{吃早饭了没?}

\textit{早饭吃了没?}

这里它传达的东西就是语义,而其格式(是否倒装)就是语法. 类似地,命题之间连接的形式就是语法,而它的真值就是语义. 因此,语义等价比语法等价(所有连接词和次序完全一致)要来得强,这是很直观的. 

有了元命题,我们才可以去说命题连接词具备某些性质:逻辑与、逻辑或具备交换律和结合律;逻辑与对逻辑或、逻辑或对逻辑与都分配;双重否定表达肯定;蕴含可以写成或和非的形式……这些东西都是清晰的,完整表列见 Rosen 的书第 29 页表 6,在此不做详述. 

接下来,很自然的一个问题是,能不能把逻辑表达式写成语义等价的某种标准形式?或者,用更加数学的语言来说,能否在语义等价划定的等价类中选取代表元,使得我们能体系化地将一个命题转化到代表元?读者不妨类比线性代数所谓的标准型,这样的思路在数学中是一脉相承的. 

现在,我们的第一个问题是:需要多少个命题连接词?因为最直观的想法就是,既然逻辑表达式的复杂性部分来源于命题连接词,那么就减少命题连接词的数量. 注意到,逻辑与可以用逻辑或和逻辑非替代,反过来逻辑或可以用逻辑与和逻辑非替代;蕴含可以用逻辑或和逻辑非替代,双向蕴含可以用蕴含和逻辑与替代. 总而言之,只需要逻辑或和逻辑非,或者逻辑与和逻辑非就可以替代掉其他所有逻辑连接词. 这种组合被称为一组\keyterm{基}[basis],而如果它能表达出所有逻辑表达式,我们就称它是\keyterm{完备的}[complete]. 但是,这里的基就没有线性代数里那么好的性质了,我们还有其他的基,可以由一个连接词构成:

\begin{example}
    定义与非:
    \[
        p \mathop{|} q :\equiv \lnot(p \wedge q)    
    \]
    它自身构成一个完备的基. 
\end{example}

\begin{proof}
    只需用它构造出非和与即可. 首先注意到
    \[
        p \mathop{|} p \equiv \lnot(p \wedge p) \equiv \lnot p
    \]
    然后与的构造无非是:
    \[
        \lnot (p \mathop{|} q) \equiv \lnot \lnot (p \wedge q) \equiv p \wedge q    
    \]
\end{proof}

\begin{exercise}
    证明或非:
    \[
        p \mathop{\downarrow} q :\equiv \lnot (p \vee q)    
    \]
    也构成一个完备的基. 
\end{exercise}

但是,哪怕只取一个单元素的基,还是会有麻烦的事情发生. 比如说下面这个等式:
\[
    [(p \mathop{|} q) \mathop{|} (p \mathop{|} q)] \mathop{|} [(p \mathop{|} q) \mathop{|} (p \mathop{|} q)] \equiv p \mathop{|} q
\]
读者不难验证它成立,事实上它只是对与非的定义式按照我们证明时的构造展开. 但是,在看到左边的式子的五秒钟内,你能发现它就是右边的式子吗?这是非常值得怀疑的. 因此,减少命题连接词的个数并不是长久之计,因为这意味着引入更多复杂的记号和等式. 

换个思路,既然 de Morgen 律告诉我们的东西是,一个复合命题的非可以被拆成两个原子命题的非,那么我们不妨\textit{只允许对原子命题取非}. 也就是说,现在我们的表达式一定不能再用 de Morgen 律分解了. 这样的原子非加上逻辑与和逻辑或显然是完备的,但是,与和或之间的分配律要求我们施加另一个限制,这个限制是稍带些对偶意味的:

\begin{definition}
    称一个复合命题是析取式(切转:合取式),如果它是由有限个原子命题或原子命题的否定的析取(切转:合取)构成的. 
\end{definition}

\begin{definition}
    称一个析取式(切转:合取式)是\keyterm{极大项}[maxterm](切转:\keyterm{极小项}[minterm]),如果其中的每个原子命题都恰出现了一次(同时出现它和它的否定算出现了两次). 
\end{definition}

\begin{definition}
    称一个命题是\keyterm{析取范式}[disjunctive normal form, DNF](切转:\keyterm{合取范式}[conjunctive normal form, CNF])的,如果它是有限个合取式(切转:析取式)的析取(切转:合取). 
\end{definition}

\begin{definition}
    称一个命题是\keyterm{主析取范式}[principal disjunctive normal form, DNF](切转:\keyterm{主合取范式}[principal conjunctive normal form, CNF])的,如果它是有限个相异的极小项(切转:极大项)的析取(切转:合取). 
\end{definition}

当然,只要对照定义就可发现,合取式和析取式都既是合取范式也是析取范式. 不妨来看一个例子:

\begin{example}
    求
    \[
        ((p \vee q) \to r) \to p
    \]
    的合取范式、析取范式和主合取范式、主析取范式. 
\end{example}

\begin{solution}
    一般的求解过程分成三步:

    第一步,我们需要消除蕴含:
    \[
        ((p \vee q) \to r) \to p \equiv \lnot ((p \vee q) \to r) \vee p \equiv \lnot (\lnot (p \vee q) \vee r) \vee p
    \]
    第二步,用 de Morgen 律和双重否定律将非全部转移到原子命题上去:
    \[
        \lnot (\lnot (p \vee q) \vee r) \vee p \equiv (\lnot \lnot (p \vee q) \wedge \lnot r) \vee p \equiv ((p \vee q) \wedge \lnot r) \vee p
    \]
    接下来的操作就按照需求的范式而定了,其主体都是来回利用分配律. 首先我们求合取范式,因为这个比较简单:
    \[
        ((p \vee q) \wedge \lnot r) \vee p \equiv (p \vee q \vee p) \wedge (\lnot r \vee p) 
    \]
    再稍作消除和补充得到主合取范式:
    \[
        (p \vee q \vee p) \wedge (\lnot r \vee p) \equiv (p \vee q) \wedge (\lnot r \vee p) \equiv (p \vee q \vee r) \wedge (p \vee q \vee \lnot r) \wedge (p \vee \lnot q \vee \lnot r) 
    \]
    接下来求析取范式:
    \[
        ((p \vee q) \wedge \lnot r) \vee p \equiv (p  \wedge \lnot r) \vee (q \wedge \lnot r) \vee p
    \]
    主析取范式则是:
    \[
        (p  \wedge q \wedge \lnot r) \vee (p  \wedge \lnot q \wedge \lnot r) \vee (p \wedge q \wedge r) \vee (p \wedge \lnot q \wedge r) \vee (\lnot p \wedge q \wedge \lnot r)
    \]
\end{solution}

注意,无论是析取范式还是合取范式,它们都不是\keyterm{典范的}[canonical]. 这意味着,存在多个满足条件的表达式. 但是,主析取范式和主合取范式是典范的,它们在置换下唯一,因此它们也被分别成为\keyterm{典范析取范式}[canonical disjunctive normal form, CDNF]和\keyterm{典范合取范式}[canonical conjunctive normal form, CCNF]. 

注意到,CDNF 和 CCNF 之间的转变其实没那么麻烦,我们重写一下上面的例子,给定 CCNF:

\begin{example}
    求
    \[
        (p \vee q \vee r) \wedge (p \vee q \vee \lnot r) \wedge (p \vee \lnot q \vee \lnot r) 
    \]
    的主析取范式. 
\end{example}

\begin{solution}
    首先注意到,求 CCNF/CDNF 的逻辑非是一个非常简单的操作,不过是把没有纳入在内的可能极大项/极小项挑出来. 因此,我们考虑对它取两次补,并利用 de Morgen 律,如下所示:
    \begin{align*}
        & (p \vee q \vee r) \wedge (p \vee q \vee \lnot r) \wedge (p \vee \lnot q \vee \lnot r)  \\
        =& \lnot \lnot ((p \vee q \vee r) \wedge (p \vee q \vee \lnot r) \wedge (p \vee \lnot q \vee \lnot r)) \\
        =& \lnot ((\lnot p \wedge \lnot q \wedge \lnot r) \vee (\lnot p \wedge \lnot q \wedge r) \vee (\lnot p \wedge q \wedge r)) \\
        =& (p  \wedge q \wedge \lnot r) \vee (p  \wedge \lnot q \wedge \lnot r) \vee (p \wedge q \wedge r) \vee (p \wedge \lnot q \wedge r) \vee (\lnot p \wedge q \wedge \lnot r)
    \end{align*}
    这和直接计算给出的结果是一致的. 
\end{solution}

\section{谓词、量词和一阶逻辑}

\section{日常生活中的命题和逻辑}

\section{推理和证明}