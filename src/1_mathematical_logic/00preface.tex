\chapter{数理逻辑:古典的和现代的}

\begin{quote}
    Wir müssen wissen, wir werden wissen.
    \flushright{David Hilbert}
\end{quote}

首先,让我们不妨先思考一个问题:什么是数学?按照古希腊人的视角,作为一种几何学的数学是通过某种\textit{公理体系}出发,然后经过一套\textit{推理规则}得出结论的过程. 那么,我们的问题有三个:

\begin{itemize}
    \item 如何选择合理的公理系统?
    \item 如何表述某种公理?
    \item 什么是合法的推理过程?
\end{itemize}

这三个问题构成了深嵌入数学基础的问题. 为了解决这种问题,命题、谓词、定义、推理这样的名词早在中世纪就已经出现,而且也有了初步的成效——事实上,早期的数学往往是建立在这样的体系之上的. 我们不妨将这个时期称为数理逻辑的古典时期,在这段时间里,虽然有不少的结果出现,但基本上都是符合直觉的——排中律、同一律、肯定前件、否定后件……这些因素几乎完全是我们日常生活中使用的规则,尽管面临经验论和先验论的论争,但没有人会去否定它们的成立性. 

如果按此种方式进行断代,第一次冲击即来源于康托的抽象公理(Axiom of Abstraction)导出的一系列令人费解的悖论,其中最有名的就是罗素悖论:

\textit{存在一个包含所有不属于自身的集合的集合。}

当然,问题的表述并不困难:这个集合是否包含其自身?如果它包含其自身,则它不应包含其自身;如果它不包含自身,则它应该包含自身。它的解决方案之一就是公理化集合论,最初的解决方式由 Zermelo 提出,他的方案成形之后也就是 Zermelo-Frankel 集合论。它通过公理建立了对集合的严格描述,进而导出了一系列有意思的结果。这个时候,连续统问题又成为了一个拦路虎:

\textit{有没有比自然数大、比实数小的集合?}

这个问题的严格提出要等到第三章阐明了关于集合的基数的概念之后,事实上,它所说的就是,是否存在一个集合,使得从自然数到它的映射可以是单射、而不能是满射,从实数到它的映射可以是满射、但不能是单射。这个问题最终引发了第二次变革,形成了“后哥德尔”的数理逻辑。

\section{学派、起源和演变}

虽然无意深究数学史,但是,对于一些基本的“主义”的了解对于感兴趣的读者来说大概会有点意思。因此,在这里,我们对其中一些代表性的学派。记住,他们的分歧只在于:什么是合法的数学对象。

\subsection{逻辑主义(logicism)}

这个学派大概可以被视作公理化思想的严格继承者,可以追溯到莱布尼兹时代。其一些著名的人物,例如戴德金、皮亚诺、弗雷格,也做出了不少有价值的成果,例如戴德金分割、皮亚诺公理体系等等。他们将数学完全归约到一套逻辑系统的推导,并认为这种逻辑推导在本体论上是中立的。这就构成了一套反柏拉图主义的数学。

他们重要的失败尝试有二。一个是弗雷格将皮亚诺算数归约到二阶逻辑,应用了下面的原则(弗雷格第五规则):

\[
    \{x | Fx\} \equiv \{x | Gx\} \iff \forall x Fx \equiv Gx    
\]

这个规则显然导致了罗素悖论的出现。另一个尝试是罗素做出的,他为了弥补这个问题,构建了类型论的层垒宇宙。他主张,我们不应该说集合属于集合,因为集合中的东西只能是“元素”。通过这样区分元素、集合、集合的集合、集合的集合的集合……他构造了一个层垒宇宙。但这为了导出算数规则则要求一个非逻辑的原则,即基本的元素是无限个的。这使得他的项目宣告失败,但类型论的思想保留了下来,并在计算机科学的研究中愈发重要。

那么,逻辑主义真的失败了吗?并不尽然。尽管在此之后经过了长达五十多年的停滞期,C. Wright 在 1983 年提出了新的解决方案。他发现,弗雷格的推导对第五规则的应用只是为了导出以下结论:

\[
    \#F = \#G \iff \exists \text{双射} f: F \mapsto G    
\]

即 $F$ 的个数与 $G$ 的个数相等,当且仅当存在将每个 $F$ 映到 $G$ 的一一对应。这被称为休谟规则。而且,在四年之后,G. Boolos 证明了这个规则是一致的,它不会导致悖论。这就引发了一种被称为新逻辑主义(neo-logicism)的思潮。但是,问题在于,休谟规则是否是一个纯粹逻辑的规则?大部分数学家主张并非如此。但是,这个命题的左侧,按照 Ø. Linnebo 的说法,只是“刻画”了命题右侧所说的东西,并不需要从经验世界中采取某些东西。类似于弗雷格第五规则和休谟规则的东西被称为抽象规则(abstraction principle),K. Fine,A. Weir 等研究者力图发展一套理论来判断它们是否是可接受的以及为什么,其中 A. Weir 从此出发,明确宣称了新逻辑主义背后的柏拉图主义底色。

\subsection{直觉主义}

\subsection{有穷主义}

\subsection{直谓主义(predicativism)}

直谓主义本来并不是一个重要的学派,20 世纪前期的数学哲学的底色主要是前面三种学派的论争。但是,直谓主义的观点在后来(约 60 年代末期)显现出了其极端重要性,因此,在此对它稍作讨论也是有益的。

直谓主义的起源事实上是罗素。他(援引庞加莱)对罗素悖论的分析,认为它的原因在于它定义了所有满足 $\lnot x \in x$ 的数学对象,而这是一个“暗指它自身”的东西,这被认为是一种恶性循环(vicious circle)。没有避免恶性循环的定义就被称作是非直谓的(impredicative),而合理的定义应该是直谓的,即其只依赖于独立于其自身的东西完成。很容易发现,这种要求对于柏拉图主义者来说绝对是不可接受的。他们要求数学概念自在的实体,而非依赖于定义的描述。

罗素为了确保直谓性给出了简单类型论(simple type theory),它直接地可以被看做是一种逻辑主义的尝试,正如我们在前文已经述及的那样。他的继承人也是著名数学家,赫尔曼·外尔,但是外尔明显更加倾向于庞加莱式的反逻辑主义视角,对直觉主义和柏拉图主义做了一些调和折中。他首先承认了自然数的自然性,然后通过一阶谓词明确了可定义性的概念。为了理解这一点,不妨看个例子:

\textit{称一个集合的闭包是所有包含这个集合的闭集的交。}

这个定义对于学过数学分析的同学来说应该是耳熟能详的。但是,注意,闭包本来也是一个闭集,所以非直谓性在这里明显地出现了。因此,我们的问题变成了:在不允许非直谓的定义的基础上,主流数学有多少命题是能够被建立的?

这个问题则变成了当代数学的一个核心的问题,代表性工作主要是 Feferman 和 Sch\"utte 独立完成的对超限基数(transfinite ordinal)的分析,克里普克-普拉特克集合论(Kripke-Platek set theory)以及 Harvey Friedman 的反数学(reverse mathematics)项目。大量构造主义的理论,例如经典的构造集合论也被认为是直谓的。

\subsection{柏拉图主义和新柏拉图主义}

\subsection{结构主义和唯名主义}

\subsection{其他:物理主义、范畴主义、多元主义……}

\section{数理逻辑的基本方法}

\section{概览}