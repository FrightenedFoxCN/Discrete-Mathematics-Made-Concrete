\documentclass{ctexbook}

\usepackage{geometry}
\geometry{
    % a4paper,
    paperheight=260mm,
    paperwidth=185mm,
    top=23mm,
    bottom=23mm,
    left=20mm,
    right=20mm,
}
% \usepackage[UTF8, heading = false, scheme = plain]{ctex} % 格式
\usepackage{ctex}
\usepackage[Bjornstrup]{fncychap} % found on Internet
\usepackage[utf8]{inputenc}
\usepackage{bm}
\usepackage{graphicx} % 添加图片
\usepackage{amsmath}
\renewcommand{\vec}[1]{\boldsymbol{#1}} % 生成粗体向量,而非带箭头的向量
\usepackage{amssymb}
\usepackage{booktabs} % Excel 导出的大表格
\usepackage{rotating}
\usepackage{extarrows}
\usepackage{enumitem}
\usepackage{xcolor}
\usepackage{multicol}
\usepackage{float}
\usepackage{mdframed}

\usepackage{tikz}
\usepackage{pgfplots}
\usetikzlibrary{cd, arrows, arrows.meta, calc, intersections, decorations.pathreplacing, patterns, decorations.markings}
\pgfplotsset{compat=1.17}

\usepackage{indentfirst}
\setlength{\parindent}{2em}

\usepackage{ntheorem}
\theoremheaderfont{\bfseries\heiti}
\theorembodyfont{\fangsong}

\usepackage{makeidx} % 名词索引
\makeindex % TODO 拼音排序

\usepackage{xparse}
% \keyterm{关键词}[英文] - 生成索引
% \keyterm*{关键词}[英文] - 不生成索引
\NewDocumentCommand{\keyterm}{smo}{%
    {\sffamily\heiti\bfseries{#2}%
    \IfNoValueF{#3}{({#3})}}%
    \IfBooleanF{#1}{\index{#2}}%
}

\usepackage{zhnumber}

% chapter 标题修改为第 * 讲
\ctexset{
    chapter={format={\centering\Huge\bfseries},name={第,讲},number=\arabic{chapter}},
    section={format={\raggedright\Large\bfseries},name={,},number={\arabic{chapter}.\arabic{section}}},
    subsection={format={\raggedright\large\bfseries},name={,},number={\arabic{chapter}.\arabic{section}.\arabic{subsection}}},
    subsubsection={format={\raggedright\normalsize\bfseries},name={,},number={\arabic{chapter}.\arabic{section}.\arabic{subsection}.\arabic{subsubsection}}},
}

\usepackage{mismath} % 包含 rank, span 等命令

\usepackage{pdfpages}

% hyperref 与 cleveref 需要最后引入
\usepackage{hyperref}
\usepackage{cleveref}


\newtheorem{definition}{定义}[chapter] % 中文
\newtheorem{example}{例}[chapter]
\newtheorem{lemma}{引理}[chapter]
\newtheorem{theorem}{定理}[chapter]
\newtheorem{corollary}{推论}[chapter]
\newenvironment{proof}{{\noindent\bfseries\heiti 证明}\quad\fangsong}{\hfill$\square$\par}
\newenvironment{solution}{\fangsong}{\par}

\renewcommand{\figureautorefname}{图}
\renewcommand{\tableautorefname}{表}
\renewcommand{\equationautorefname}{式}
\renewcommand{\theoremautorefname}{定理}
\renewcommand{\sectionautorefname}{节}
\newcommand{\lemmaautorefname}{引理}
\newcommand{\corollaryautorefname}{推论}
\newcommand{\exampleautorefname}{例}
\newcommand{\definitionautorefname}{定义}
\crefrangeformat{equation}{式~#3#1#4--#5#2#6}
\crefrangeformat{example}{例~#3#1#4--#5#2#6}

\title{Discrete Mathematics Made Concrete}
\author{FrightenedFoxCN \quad frightenedfox@outlook.com}

% 嵌套 enumerate 环境的 label
\setlist[enumerate,2]{label=(\arabic*)}
\setlist[enumerate,3]{label=\roman*.}

\begin{document}
\frontmatter

\maketitle

\null
\thispagestyle{empty}
\clearpage

\pdfbookmark[0]{目录}{contents}
\tableofcontents

\addtolength{\parskip}{.5em}

\mainmatter
\setcounter{page}{1} % 将页码计数设置为 1
\chapter{概率:随机性的算法应用}
\chapter{古典逻辑学的观念}

\begin{quote}
    Each mortal thing does one thing and the same: \\
    Deals out that being indoors each one dwells; \\
    Selves - goes itself; myself it speaks and spells, \\
    Crying What I do is me: for that I came.
\flushright{G. M. Hopkins \textit{As kingfishers catch fire}}
\end{quote}

古典逻辑学的历史我们已经谈了不少了,接下来我们需要谈谈它实质性的内容。当然,这些内容几乎都是近乎常识,但是 Rosen 的书中还是把它专划作一个章节讨论。因此,我们在此也不妨将其单独拎出来作为一讲。这里主要讨论的是命题、逻辑连接词、量词、谓词和范式的问题,同时,真值表的处理方式也被放在这里。

\section{命题}

什么是\keyterm{命题}[proposition]?高中的时候我们就知道,命题是一个或真或假的陈述性语句。比方说,“太阳东升西落”就是一个命题,而“存在一个包含所有不属于自身的集合的集合”就不是一个命题。当然,后半句话有点拿当朝的剑斩前朝的官的意思,因为这个悖论在古典逻辑盛行的时候并没有出现,但既然 Rosen 的书上这么定义,而且有些老师也会这样举例,那我们就姑且承认这不是一个命题。

然后,我们用类似语言哲学家们的方式来处理\keyterm{逻辑连结词}[connective]非:

\begin{definition}
    设 $p$ 为一命题,则 $\lnot p$ 表示命题“情况并非如此,$p$”(it is not the case that...)。
\end{definition}

$\lnot p$ 被称为命题 $p$ 的\keyterm{否定}[negation]。“与”和“或”也用类似的方式定义:

\begin{definition}
    设 $p, q$ 为命题,则 $p \wedge q$ 表示 $p$ 且 $q$,即当且仅当 $p$ 和 $q$ 均为真时它才为真;$p \vee q$ 表示 $p$ 或 $q$,即当且仅当 $p$ 和 $q$ 中至少有一个为真时它才为真。
\end{definition}

$p \wedge q$ 也被称为 $p$ 和 $q$ 的\keyterm{合取}[conjunction],$p \vee q$ 则被对应的称为 $p$ 和 $q$ 的\keyterm{析取}[disjunction]。单个的命题 $p$ 和 $q$ 被称为\keyterm{原子命题}[atomic proposition],它被认为是不可再分的;而由逻辑连接词连接起来的命题则被称为\keyterm{复合命题}[compound proposition]。显然,面对实际问题时,原子命题的选择具备某种程度上的任意性。逻辑连接词又称\keyterm{逻辑运算符}[logical operator]。

现在看来除了术语已经开始复杂化了以外都很顺利,但是,这种连接词的表达方式多少有点不够形式化。因此,我们引入\keyterm{真值表}[truth table]的概念。

\begin{table}[htpb]
    \caption{逻辑非的真值表}
    \begin{center}
    \begin{tabular}{c|c}
    \hline
    $p$ & $\lnot p$ \\ \hline
    $T$ & $F$       \\ \hline
    $F$ & $T$       \\ \hline
    \end{tabular}
    \end{center}
\end{table}

\begin{table}[htpb]
    \caption{逻辑与的真值表}
    \begin{center}
    \begin{tabular}{c|c|c}
    \hline
    $p$ & $q$ & $p \wedge q$ \\ \hline
    $T$ & $T$ & $T$          \\ \hline
    $T$ & $F$ & $F$          \\ \hline
    $F$ & $T$ & $F$          \\ \hline
    $F$ & $F$ & $F$          \\ \hline
    \end{tabular}
    \end{center}
\end{table}

\begin{table}[htpb]
    \caption{逻辑或的真值表}
    \begin{center}
    \begin{tabular}{c|c|c}
    \hline
    $p$ & $q$ & $p \vee q$ \\ \hline
    $T$ & $T$ & $T$          \\ \hline
    $T$ & $F$ & $T$          \\ \hline
    $F$ & $T$ & $T$          \\ \hline
    $F$ & $F$ & $F$          \\ \hline
    \end{tabular}
    \end{center}
\end{table}

其中左边的一列/两列称为对命题的\keyterm{赋值}[assignment/evaluation],后边一列则是逻辑连接符运算的结果,$T$ 表示真,$F$ 表示假。这样,我们就能定义一些更加复杂的连接词:

\begin{definition}
    设 $p, q$ 为命题,则 $p \oplus q$ 表示 $p$ 和 $q$ 的\keyterm{异或}[exclusive or],它为真当且仅当 $p$ 和 $q$ 中有且仅有一个为真。
\end{definition}

\begin{table}[htpb]
    \caption{异或的真值表}
    \begin{center}
    \begin{tabular}{c|c|c}
    \hline
    $p$ & $q$ & $p \oplus q$ \\ \hline
    $T$ & $T$ & $F$          \\ \hline
    $T$ & $F$ & $T$          \\ \hline
    $F$ & $T$ & $T$          \\ \hline
    $F$ & $F$ & $F$          \\ \hline
    \end{tabular}
    \end{center}
\end{table}

\begin{definition}
    设 $p, q$ 为命题,则 $p \to q$ 表示若 $p$ 则 $q$,它为真当且仅当 $p$ 为假或者 $p$ 和 $q$ 均为真。这种式子被称为\keyterm{蕴含式}[implication]或\keyterm{条件句}[conditional statement],其中 $p$ 称为\keyterm{前提}[premise]、\keyterm{假设}[hypothesis]或\keyterm{前项}[antecedent],$q$ 称为\keyterm{结论}[conclusion]或\keyterm{结果}[consequence]。
\end{definition}

\begin{table}[htpb]
    \caption{蕴含的真值表}
    \begin{center}
    \begin{tabular}{c|c|c}
    \hline
    $p$ & $q$ & $p \to q$ \\ \hline
    $T$ & $T$ & $T$          \\ \hline
    $T$ & $F$ & $F$          \\ \hline
    $F$ & $T$ & $T$          \\ \hline
    $F$ & $F$ & $T$          \\ \hline
    \end{tabular}
    \end{center}
\end{table}

这里有一个有趣的问题:当前提为假的时候,结论不管怎么说都是真的。这看起来反直觉,但其实是一个很平常的事情,看下面这句话:

\textit{如果你好好学习,你就不会挂科。}

这句话是真的,哪怕你没有好好学习然后挂科了,或者没有好好学习但没挂科。前提不成立的情况下,任何事实结果都不会影响这个命题是否是真的。在下面这个对话中,这个性质就表现得更加明显了:

\textit{``Can you v me 50?" ``When pigs fly!"}

第二个说话者的意思事实上是绝无此可能。因为猪会飞是一个假命题,所以它使得被省略的命题(\textit{``I can v you 50"})一定成立,但这是反事实的。这就构成了一个重要的逻辑原则,它被称为 ex falso sequitur quodlibet:从谬误中,你想怎么干就怎么干,形式化地说,对于任意命题 $p$,$F \to p$ 成立。

最后一个逻辑运算符比较简单:

\begin{definition}
    设 $p, q$ 为命题,则 $p \leftrightarrow q$ 表示 $p$ 当且仅当 $q$,即两者具备相同的真值。我们将其称为\keyterm{双向蕴含式}[biconditional statements/bi-implications]
\end{definition}

\begin{table}[htpb]
    \caption{双向蕴含的真值表}
    \begin{center}
    \begin{tabular}{c|c|c}
    \hline
    $p$ & $q$ & $p \leftrightarrow q$ \\ \hline
    $T$ & $T$ & $T$          \\ \hline
    $T$ & $F$ & $F$          \\ \hline
    $F$ & $T$ & $F$          \\ \hline
    $F$ & $F$ & $T$          \\ \hline
    \end{tabular}
    \end{center}
\end{table}

我们约定,逻辑运算符的优先级为 $\lnot$ 最高,$\wedge$ 和 $\vee$ 其次,$\to$ 和 $\leftrightarrow$ 最低。因为 $\oplus$ 基本上不会出现,所以我们也可以预设它的有限集和 $\vee$ 一致,但这没什么影响。

在这一小节最后,我们讲讲怎么利用真值表推算复合命题的真值。下面是一个例子:

\begin{table}[htpb]
    \caption{$(p \vee \lnot q) \to (p \wedge q)$ 的真值表}
    \begin{center}
    \begin{tabular}{c|c|c|c|c|c}
    \hline
    $p$ & $q$ & $\lnot q$ & $p \vee \lnot q$ & $p \wedge q$ & $(p \vee \lnot q) \to (p \wedge q)$ \\ \hline
    $T$ & $T$ & $F$ & $T$ & $T$ & $T$ \\ \hline
    $T$ & $F$ & $T$ & $T$ & $F$ & $F$\\ \hline
    $F$ & $T$ & $F$ & $F$ & $F$ & $T$ \\ \hline
    $F$ & $F$ & $T$ & $T$ & $F$ & $F$ \\ \hline
    \end{tabular}
    \end{center}
\end{table}

也就是说,只要逐级把表达式拆开,然后从小到大计算。这样的思路在后面讨论数理逻辑的问题也会碰到,这里其实有一个被称为\keyterm{结构归纳}[structural induction]的方法,留待后面才会得到真正意义上的广泛应用。

\section{日常生活中的命题和逻辑}

\section{命题的等性以及范式}

\section{谓词、量词和一阶逻辑}

\section{推理和证明}

% \setcounter{chapter}{0}
% \chapter{概率:随机性的算法应用}

% \setcounter{chapter}{0}
% \chapter{概率:随机性的算法应用}

% \setcounter{chapter}{0}
% \chapter{概率:随机性的算法应用}

% \setcounter{chapter}{0}
% \chapter{概率:随机性的算法应用}

% \setcounter{chapter}{0}
% \chapter{概率:随机性的算法应用}

% \setcounter{chapter}{0}
% \chapter{概率:随机性的算法应用}

% \setcounter{chapter}{0}
% \chapter{概率:随机性的算法应用}

% \setcounter{chapter}{0}
% \chapter{概率:随机性的算法应用}

\backmatter
\pdfbookmark[0]{索引}{index}
\printindex

\end{document}